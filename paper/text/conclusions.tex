%!TEX root = ../authorinstr.tex

\section{Conclusions and further work}
Two different environments are used to compare the performance of the three different RL algorithms. ... performed best in the MountainCar environment obtaining ... Results indicate that ... performs best in the LunarLander environment providing a ... To conclude, SARSA/CACLA/NFAC provides the best results using the parameters ... in MountainCar whereas SARSA/CACLA/NFAC with ... gives the best results for LunarLander. This AGREES/DISAGREES with previous expectations that were stated earlier. However one should note that more data needs to be collected to be more certain about the validity of the results, but due to time restrictions a limited amount of trials are rendered in this research.  

Further work needs to be done in testing different input representations. The OpenAI environments provide an initial state and a new state after every action performed. One can alter the state representation before it is used by the function approximator. An attempted input representation was representing each state value by N units. Each unit is responsible for a range of values; e.g. the range [-1, 1] can be represented in steps of 0.1 by choosing $N=20$ units. Then the value -0.85 is represented by the second hidden unit, so this value is set to 1. The two neighbouring states are set to 0.5 to keep the input representation continuous. This is done for each state value and the result is a concatenated vector holding the new state. Another attempted approach is similar as before, but instead of setting the neighbouring states to 0.5, a convolution is performed with a gaussian window. This way the input can approximate a gaussian even better. Results obtained with these approaches were not evidently better on a first glance. However this should be extensively investigated in further research. Additionally more research should be done with other environments, to be able to generalize more about the performance of Neural Fitted Actor-Critic.

