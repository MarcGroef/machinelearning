%!TEX root = ../authorinstr.tex

\section{Experimental setup}


In this section, the used parameters are described as used in the experiments. We ran each setting twice. The settings where the goal was reached at least 1 time withing the last 100 epochs where run 5 times for significance. For each algorithm in each environment the best setting was chosen and is shown in table \ref{tab:mntparam} and table \ref{tab:lunarparam} for MountainCar and LunarLander respectively. The scores obtained with these settings were used for comparrison of the algorithms. 

For all three algorithms, the explored learningrates were 0.05, 0.01, 0.001. For the discount parameter, the following values were used: 0.999, 0.99, 0.9, 0.8 0.6 0.4 0.2. The number of nodes used in the MLP(s) used were 20, 50.

\subsection{GD-SARSA}
GD-SARSA used $\epsilon$-greedy exploration, which was first set to 1.0, resulting initially in a fully pseudo-random behaviour. This exploration rate decayed over epochs, and is equal to $0.99^{N-1}$, where N is the current epoch. Since the experiments run for 2000 epochs, this means that the final exploration rate is $1.86*10^{-9}$. For each action selection, 10 iterations of gradient descend are used to obtain a better action. 
\subsection{CACLA}
CACLA used Gaussian exploration, where a random value sampled from a gaussian distribution($\mu=0$ $\sigma=1$) was multiplied by a value $\Sigma=10$. This value was added to the output of the MLP and finally clamped to be in the range [$-1$,$1$]. This $\Sigma$ decayed over epochs and is equal to  $10 * 0.99^{N-1}$, where N is the current epoch. Since the experiments run for 2000 epochs, this means that the final exploration rate is $1.86*10^{-8}$. Since initially the exploration rate is very high, this ensures that the agent will quickly reach its goal. 
%TODO: note constant parameters, and params swept over
\subsection{NFAC}
NFAC used an identical exploration method to CACLA.
%TODO: note constant parameters, and params swept over

%Should be divided to CACLA and NFAC.
%NFAC and CACLA both used gaussian exploration.  A random value sampled from a gaussian distribution($\mu=0$ $\sigma=1$) was multiplied by a value $\Sigma=10$. This value was added to the output of the MLP and finally clamped to be in the range [$-1$,$1$]. This $\Sigma$ decayed over epochs and is equal to  $10 * 0.99^{N-1}$, where N is the current epoch. Since the experiments run for 2000 epochs, this means that the final exploration rate is $1.86*10^{-8}$. Since initially the exploration rate is very high, this ensures that the agent will quickly reach its goal. [[The exploration rate is diminshed over time since then it can rely more on its learned behavior rather than the noise added by the gaussion value.]] 

\begin{table}
\centering
\label{tab:mntparam}
\begin{tabular}{r|llll}
                     & learning rate & discount factor & number of hidden nodes \\\hline
SARSA & N          & P               & M         \\
CACLA & A          & B               & actor: F, critic: G         \\
NFAC    & X          & Y              & actor: H, critic: I        
\end{tabular}
\caption{MountainCar network parameters per algorithm}
\end{table}

\begin{table}
\centering
\label{tab:lunarparam}
\begin{tabular}{r|llll}
                     & learning rate & discount factor & number of hidden nodes \\\hline
SARSA & N          & P               & M         \\
CACLA & A          & B               & actor: F, critic: G         \\
NFAC    & X          & Y              & actor: H, critic: I        
\end{tabular}
\caption{LunarLander network parameters per algorithm}
\end{table}

